\documentclass[../projekt.tex]{subfiles}
\begin{document}


\chapter{Uživatelská příručka}

Tato aplikace slouží k demonstraci Kargerova algoritmu. 

\section{Ovládání aplikace}


	\begin{figure}[ht]
    	\begin{center}
  			\includegraphics[scale=0.36]{obrazky-figures/ovladani.png}
  			\caption{Rozhraní ovládání aplikace.}
  		\end{center}
	\end{figure}
	
Aplikaci lze ovládat pomocí:
	
\begin{itemize}
	\item MENU - Práce se souborem, nápověda.
	\item EDITORU - Úprava grafu.
	\item OVLÁDACÍHO PANELU - Aplikace algoritmu na graf.
\end{itemize}




\subsection{Menu}

\begin{itemize}
    \item \includegraphics[height=10em]{obrazky-figures/file.png} \\Možnosti práce se souborem: \\ - vytvoření nového grafu \\ - načtení uloženého grafu ve formátu XML \\ - uložení grafu ve formátu XML
    \item \includegraphics[height=10em]{obrazky-figures/help.png} \\Možnosti nápovědy: \\ - uživatelská příručka \\ - o aplikaci \\
\end{itemize}



\subsection{Editor}

Editor obsahuje seznam uzlů (Node list) a seznam hran (Edge list). Ke každému z těchto seznamů jsou přidružena dvě tlačítka: 

\begin{itemize}
    \item \includegraphics[height=1.3em]{obrazky-figures/addButton.png} - Vložení uzlu/hrany do grafu
    \item \includegraphics[height=1.3em]{obrazky-figures/removeButton.png} - Odebrání uzlu/hrany z grafu \\
\end{itemize} 



\subsection{Ovládací panel}

Ovládací panel obsahuje pět různých tlačítek, přičemž aplikace může běžet ve dvou různých módech. První tlačítko zleva je:  

\begin{itemize}
    \item \includegraphics[height=1.3em]{obrazky-figures/resetSmall.png} - Reset grafu. Vrátí graf do počátečního stavu a umožňuje začít znovu. 
\end{itemize}

\newpage

\noindent Zbývající čtyří tlačítka pracují v různých módech aplikace: \\

\noindent Tzv. Single run mód:

\begin{itemize}
    \item \includegraphics[height=1.3em]{obrazky-figures/playSmall.png} - Kliknutím na toto tlačítko dojde k provedení jednoho kroku algoritmu, tedy ke sloučení jedné dvojice uzlů a zobrazení aktualizovaného grafu.
    \item \includegraphics[height=1.3em]{obrazky-figures/nextStepSmall.png} - Kliknutím na toto tlačítko dojde ke spuštění jednoho běhu algoritmu a zobrazení výsledku tohoto běhu.
    \item \includegraphics[height=1.3em]{obrazky-figures/manualSteps.png} - Kliknutím na toto tlačítko je aktivováno krokování algoritmu v pravém panelu a zobrazování změn na grafu. 
\end{itemize}

\noindent Tzv. Multiple runs mód:

\begin{itemize}
	\item \includegraphics[height=1.3em]{obrazky-figures/finishSmall.png} - Kliknutím na dané tlačítko je spuštěn kompletní běh algoritmu. Algoritmus tedy proběhne pro všechny možné varianty. Po dokončení zobrazí graf v počátečním stavu, nejlepší možný výsledek a všechny další výsledky.
\end{itemize}

\end{document}

